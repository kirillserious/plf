\documentclass[a4paper, 11pt]{article}

% Внешние пакеты
\usepackage{amsmath}
\usepackage{amssymb}
\usepackage{hyperref}
\usepackage{url}
\usepackage{a4wide}
\usepackage[utf8]{inputenc}
\usepackage[main = russian, english]{babel}
\usepackage[pdftex]{graphicx}
\usepackage{float}
\usepackage{subcaption}
\usepackage{indentfirst}
\usepackage{amsthm}                  % Красивый внешний вид теорем, определений и доказательств
% \usepackage[integrals]{wasysym}    % Делает интегралы прямыми, но некрасивыми
% \usepackage{bigints}               % Позволяет делать большущие интегралы

% Красивый внешний вид теорем, определений и доказательств
\newtheoremstyle{def}
        {\topsep}
        {\topsep}
        {\normalfont}
        {\parindent}
        {\bfseries}
        {.}
        {.5em}
        {}
\theoremstyle{def}
\newtheorem{definition}{Определение}[section]
\newtheorem{example}{Пример}[section]

\newtheoremstyle{th}
        {\topsep}
        {\topsep}
        {\itshape}
        {\parindent}
        {\bfseries}
        {.}
        {.5em}
        {}
\theoremstyle{th}
\newtheorem{theorem}{Теорема}[section]
\newtheorem{lemma}{Лемма}[section]
\newtheorem{assertion}{Утверждение}[section]

\newtheoremstyle{rem}
        {0.5\topsep}
        {0.5\topsep}
        {\normalfont}
        {\parindent}
        {\itshape}
        {.}
        {.5em}
        {}
\theoremstyle{rem}
\newtheorem{remark}{Замечание}[section]

% Новое доказательство
\renewenvironment{proof}{\parД о к а з а т е л ь с т в о.}{\hfill$\blacksquare$}

% Нумерация рисунков и уравнений
\numberwithin{figure}{section}
\numberwithin{equation}{section}

% Переопределение математических штук
        \DeclareMathOperator{\sgn}{sgn}
        \DeclareMathOperator{\const}{const} 
        \newcommand{\T}{\mathrm{T}}         % Транспонирование
        % Множества
        \newcommand{\SetN}{\mathbb{N}}
        \newcommand{\SetZ}{\mathbb{Z}}
        \newcommand{\SetQ}{\mathbb{Q}}
        \newcommand{\SetR}{\mathbb{R}}
        \newcommand{\SetC}{\mathbb{C}}
        % Теорвер
        \newcommand{\Prb}{\mathbb{P}}
        \newcommand{\Ind}{\mathbb{I}}
        \newcommand{\Exp}{\mathbb{E}}
        \newcommand{\Var}{\mathbb{V}\mathrm{ar}}
        % Оптималка
        \newcommand{\SetX}{\mathcal{X}}
        \newcommand{\SetP}{\mathcal{P}}

% Дополнительные штуки
        \usepackage[left=2cm,right=2cm,top=2cm,bottom=2cm,bindingoffset=0cm]{geometry}
\usepackage[T2A]{fontenc}
\usepackage{units}
\usepackage{arcs}

\newtheorem{lemm}{Лемма}
\newcommand{\scalar}[2]{\left<#1,#2\right>}
\newcommand{\norm}[1]{\left\lVert #1 \right\rVert}
\newcommand{\pd}[2]{\dfrac{\partial{#1}}{\partial{#2}}}
\newcommand{\R}{\mathbb{R}}
\newcommand{\Z}{\mathbb{Z}}
\newcommand{\Cn}{\mathbb{C}}
\renewcommand{\C}{\mathrm{const}}
\renewcommand{\le}{\leqslant}
\renewcommand{\ge}{\geqslant}
\DeclareMathOperator*{\Res}{Res}
\renewcommand{\Im}{\mathrm{Im}\,}
\renewcommand{\Re}{\mathrm{Re}\,}

        \def\hevKul{\chi(t)}
\def\lftKul{\supset}
\def\bsbKul{\boldsymbol }

        \usepackage[left=2cm,right=2cm,top=2cm,bottom=2cm,bindingoffset=0cm]{geometry}
\usepackage[T2A]{fontenc}
\usepackage{units}
\usepackage{arcs}

\newtheorem{lemm}{Лемма}
\newcommand{\scalar}[2]{\left<#1,#2\right>}
\newcommand{\norm}[1]{\left\lVert #1 \right\rVert}
\newcommand{\pd}[2]{\dfrac{\partial{#1}}{\partial{#2}}}
\newcommand{\R}{\mathbb{R}}
\newcommand{\Z}{\mathbb{Z}}
\newcommand{\Cn}{\mathbb{C}}
\renewcommand{\C}{\mathrm{const}}
\renewcommand{\le}{\leqslant}
\renewcommand{\ge}{\geqslant}
\DeclareMathOperator*{\Res}{Res}
\renewcommand{\Im}{\mathrm{Im}\,}
\renewcommand{\Re}{\mathrm{Re}\,}

        \usepackage[left=2cm,right=2cm,top=2cm,bottom=2cm,bindingoffset=0cm]{geometry}
\usepackage[T2A]{fontenc}
\usepackage{units}
\usepackage{arcs}

\newtheorem{lemm}{Лемма}
\newcommand{\scalar}[2]{\left<#1,#2\right>}
\newcommand{\norm}[1]{\left\lVert #1 \right\rVert}
\newcommand{\pd}[2]{\dfrac{\partial{#1}}{\partial{#2}}}
\newcommand{\R}{\mathbb{R}}
\newcommand{\Z}{\mathbb{Z}}
\newcommand{\Cn}{\mathbb{C}}
\renewcommand{\C}{\mathrm{const}}
\renewcommand{\le}{\leqslant}
\renewcommand{\ge}{\geqslant}
\DeclareMathOperator*{\Res}{Res}
\renewcommand{\Im}{\mathrm{Im}\,}
\renewcommand{\Re}{\mathrm{Re}\,}
